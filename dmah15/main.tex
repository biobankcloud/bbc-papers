% IEEE Paper Template for US-LETTER Page Size (V1)
% Sample Conference Paper using IEEE LaTeX style file for US-LETTER pagesize.
% Copyright (C) 2006-2008 Causal Productions Pty Ltd.
% Permission is granted to distribute and revise this file provided that
% this header remains intact.
%
% REVISION HISTORY
% 20080211 changed some space characters in the title-author block
%
% \documentclass[10pt,conference,letterpaper]{IEEEtran}
% \usepackage{times,amsmath,epsfig}
% \usepackage[utf8]{inputenc}
% \usepackage[usenames,dvipsnames]{color}
% \usepackage{amssymb}

\documentclass[a4paper,english]{llncs}
\usepackage[T1]{fontenc}
\usepackage[latin9]{inputenc}
\usepackage{float}
\usepackage{slashed}
\usepackage{graphicx}
\usepackage{fixltx2e}
\usepackage{amsmath}
\usepackage{url}
\usepackage{seqsplit} % for splitting very long variable names
\makeatletter
\raggedbottom % remove unwanted space between paragraphs 

%%%%%%%%%%%%%%%%%%%%%%%%%%%%%% LyX specific LaTeX commands.
\pdfpageheight\paperheight
\pdfpagewidth\paperwidth

\floatstyle{ruled}
\newfloat{algorithm}{tbp}{loa}
\providecommand{\algorithmname}{Algorithm}
\floatname{algorithm}{\protect\algorithmname}

%%%%%%%%%%%%%%%%%%%%%%%%%%%%%% User specified LaTeX commands.
\usepackage[noend]{algorithm,algcompatible,algpseudocode}
\usepackage{amssymb} % for symbols 
\usepackage{caption} % algorthm is a float and does not split. use caption instead
\usepackage{subcaption}
\usepackage{etoolbox}\AtBeginEnvironment{algorithmic}{\footnotesize}
%\algsetup{linenosize=\small}
\usepackage{subfig}
\makeatother
\algblockdefx[NAME]{StartTransaction}{EndTransaction}
[1]{\textbf{begin transaction} \textbf{#1}}{\textbf{commit transaction}}
\algrenewcommand\alglinenumber[1]{\footnotesize #1:}

\usepackage{babel}
\usepackage{cite}
\usepackage{amssymb}% http://ctan.org/pkg/amssymb
\usepackage{pifont}% http://ctan.org/pkg/pifont
\newcommand{\cmark}{\ding{51}}%
\newcommand{\xmark}{\ding{53}}%

\title{BiobankCloud}

\author{Jim Dowling \inst{1,2}}

\institute{KTH - Royal Institute of Technology,\\
\email{\{jdowling\}@kth.se}
\and
SICS - Swedish ICT \\
\email{\{jdowling\}@sics.se}
}

\newif\ifshowcomments
\showcommentstrue

\ifshowcomments
\newcommand{\mynote}[2]{\fbox{\bfseries\sffamily\scriptsize{#1}}
{\small$\blacktriangleright$\textsf{\emph{#2}}$\blacktriangleleft$}}
\else
\newcommand{\mynote}[2]{}
\fi
\newcommand{\jim}[1]{\textcolor{Red}{\mynote{Jim}{#1}}}

\begin{document}
\maketitle

\begin{abstract}

\end{abstract}



\section{Introduction}
Biobanks store and catalog human biological material from identifiable individuals for both clinical and research purposes. Recent initiatives in personalized medicine created a steeply increasing demand to sequence the human biological material stored in biobanks. As of 2015, such large-scale sequencing is under way in hundreds of projects around the world, with the largest single project sequencing up to 100.000 genomes\footnote{See \url{http://www.genomicsengland.co.uk/}.}. Furthermore, sequencing also is becoming more and more routine in a clinical setting for improving diagnosis and therapy especially in cancer\cite{pmedicine}. However, software systems for biobanks traditionally managed only metadata associated with samples, such as pseudo-identifiers for patients, sample collection information, or study information. Such systems cannot cope with the current requirement to, alongside such metadata, also store and analyze genomic data, which might mean everything from a few Megabytes (e.g., genotype information from a SNP array) to hundreds of Gigabytes per sample (for whole genome sequencing with high coverage). 

For a long time, such high-throughput sequencing and analysis was only available to large research centers that (a) could afford enough modern sequencing devices and (b) had the budget and IT expertise to manage high performance computing clusters. This situation is changing. The cost of sequencing is falling rapidly, and more and more labs and hospitals depend on sequencing information for daily research and diagnosis/treatment. However, there is still a pressing need for flexible and open software systems to enable the computational analysis of large biomedical data sets at a reasonable price. Note that this trend is not restricted to genome sequencing; very similar developments are also happening in other medical areas, such as molecular imaging\cite{imaging}, drug discovery\cite{drug}, or data generated from patient-attached sensors\cite{qself}. 

In this paper, we present the BiobankCloud platform, a collaborative project bringing together computer scientists, bioinformaticians, pathologists, and biobankers. The system is designed as a ``platform-as-a-service'', i.e., it can be easily installed on a local cluster (or, equally well, in a public cloud) using Karamel and Chef\footnote{\url{http://www.karamel.io}}. Primary design goals are flexibility in terms of the analysis being performed, scalability up to very large data sets and very large cluster set-ups, ease of use and low maintenance cost, strong support for data security and data privacy, and direct usability for users. To this end, it encompasses (a) a scientific workflow engine running on top of the popular Hadoop platform for distributed computing, (b) a scientific workflow language focusing on easy integration of existing tools and simple rebuilding of existing pipelines, (c) support for automated installation, and (d) role-based access control. It also features (e) HopsFS, a new version of Hadoop's Distributed Filesystem (HDFS) with improved throughput, supported for extended metadata, and reduced storage requirements compared to HDFS, (f) Charon, which enables the federation of clouds at the file system level, and (g) a simple Laboratory Information Management Service with an integrated web interface for authenticating/authorizing users, managing data, designing and searching for metadata, and support for running workflows and analysis jobs on Hadoop. This web interface hides much of the complexity of the Hadoop backend, and supports multi-tenancy through first-class support for \textit{Studies}, \textit{SampleCollections} (DataSets), \textit{Samples}, and \textit{Users}. 

In this paper, we give an overview on the architecture of the BiobankCloud platform and describe each component in more detail. The system is currently under development; while a number of components already have been released for immediate usage (e.g., Hops, SAASFEE), a first overall platform release is planned for the near future. The system is essentially agnostic to the type of data being managed and the types of analysis being performed, but developed with genome sequencing as most important application area. Therefore, throughout this paper we will use examples from this domain. 


\section{Related work}
There are a couple of frameworks for data parallel processing of genomic data. Adam, Halvade, Seal, PigSeq, Spork?

There's the big Alvados? project in Harvard.

In security, Hadoop has support for Apache Ranger (attribute-based access control), Apache Sentry (Databases, RBAC), and Apache ?? (RBAC for REST APIs).

Nothing happening in Biobanking?



\section{Conclusions}
We have shown that a reliable leader election service can be implemented using two phase commit transactions in the NDB storage engine, a NewSQL database. Our solution ensures that there is never more than one leader, and the time taken for leader election is comparable to ZooKeeper for clusters of up to 800 processes. Our algorithm enables distributed systems that already use NewSQL databases to save the operational overhead of deploying a third-party service, such as ZooKeeper, for leader election, as our algorithm can easily be re-implemented for other NewSQL databases. 
\section{Acknowledgements}
This work funded by the EU FP7 project ``Scalable, Secure Storage and Analysis of Biobank Data'' under Grant Agreement no. 317871. 
\bibliographystyle{ieeetr}
\bibliography{bbc}

\end{document}
