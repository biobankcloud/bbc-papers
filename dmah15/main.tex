% IEEE Paper Template for US-LETTER Page Size (V1)
% Sample Conference Paper using IEEE LaTeX style file for US-LETTER pagesize.
% Copyright (C) 2006-2008 Causal Productions Pty Ltd.
% Permission is granted to distribute and revise this file provided that
% this header remains intact.
%
% REVISION HISTORY
% 20080211 changed some space characters in the title-author block
%
% \documentclass[10pt,conference,letterpaper]{IEEEtran}
% \usepackage{times,amsmath,epsfig}
% \usepackage[utf8]{inputenc}
% \usepackage[usenames,dvipsnames]{color}
% \usepackage{amssymb}

\documentclass[a4paper,english]{llncs}
\usepackage[T1]{fontenc}
\usepackage[latin9]{inputenc}
\usepackage{float}
\usepackage{slashed}
\usepackage{graphicx}
\usepackage{fixltx2e}
\usepackage{amsmath}
\usepackage{url}
\usepackage{seqsplit} % for splitting very long variable names
\makeatletter
\raggedbottom % remove unwanted space between paragraphs 

%%%%%%%%%%%%%%%%%%%%%%%%%%%%%% LyX specific LaTeX commands.
\pdfpageheight\paperheight
\pdfpagewidth\paperwidth

\floatstyle{ruled}
\newfloat{algorithm}{tbp}{loa}
\providecommand{\algorithmname}{Algorithm}
\floatname{algorithm}{\protect\algorithmname}

%%%%%%%%%%%%%%%%%%%%%%%%%%%%%% User specified LaTeX commands.
\usepackage[noend]{algorithm,algcompatible,algpseudocode}
\usepackage{amssymb} % for symbols 
\usepackage{caption} % algorthm is a float and does not split. use caption instead
\usepackage{subcaption}
\usepackage{etoolbox}\AtBeginEnvironment{algorithmic}{\footnotesize}
%\algsetup{linenosize=\small}
\usepackage{subfig}
\makeatother
\algblockdefx[NAME]{StartTransaction}{EndTransaction}
[1]{\textbf{begin transaction} \textbf{#1}}{\textbf{commit transaction}}
\algrenewcommand\alglinenumber[1]{\footnotesize #1:}

\usepackage{babel}
\usepackage{cite}
\usepackage{amssymb}% http://ctan.org/pkg/amssymb
\usepackage{pifont}% http://ctan.org/pkg/pifont
\newcommand{\cmark}{\ding{51}}%
\newcommand{\xmark}{\ding{53}}%

\title{BiobankCloud}

\author{Jim Dowling \inst{1,2}}

\institute{KTH - Royal Institute of Technology,\\
\email{\{jdowling\}@kth.se}
\and
SICS - Swedish ICT \\
\email{\{jdowling\}@sics.se}
}

\newif\ifshowcomments
\showcommentstrue

\ifshowcomments
\newcommand{\mynote}[2]{\fbox{\bfseries\sffamily\scriptsize{#1}}
{\small$\blacktriangleright$\textsf{\emph{#2}}$\blacktriangleleft$}}
\else
\newcommand{\mynote}[2]{}
\fi
\newcommand{\jim}[1]{\textcolor{Red}{\mynote{Jim}{#1}}}

\begin{document}
\maketitle

\begin{abstract}

\end{abstract}


\section{Introduction}
Biobanks store and catalog human biological material from identifiable individuals for both clinical and research purposes. Recent advances in Next-Generation Sequencing (NGS) technology has meant that there is an increasing demand to sequence the human biological material stored in Biobanks. Both research projects and clinical health-care systems use Biobanks to store samples that are sequenced using techniques from genotyping to whole-genome sequencing (WGS). Biobanks' computer systems have traditionally managed only metadata associated with samples, such as pseudo-identifiers for patients, sample collection information, study infromation, and data concerning samples. Alongside this metadata, we now increasingly need to store genomic data, which requires anything from MBs (genotyping) up to several hundred GBs (WGS) of data per sample.
The data storage requirements for large-scale WGS sequencing projects, such as the 100,000 genomes project by Genomics England, now scale to many PBs, and are so large that researchers are looking at cost-effective Big Data solutions based on commodity hardware, such as Hadoop. When data volumes grow past several TBS, traditional database technology (including sharded relational databases) is no longer viable, as more data needs to be moved from storage to compute nodes that is possible with current networking technology. The solution provided by platforms such as Hadoop is to move computation to where the data is located, exploiting the principle of \textit{data locality}. That is, jobs are parallelized across many nodes and each job loads its (large) input data primarily from disk subsystems, while network I/O is used transfer in subsequent processing and sorting steps on, typically, smaller data volumes relative to the input data size.

In this paper, we introduce BiobankCloud, an integrated platform, based on Hadoop, for the secure storage, processing, and sharing of genomic data and associated metadata. BiobankCloud is a platform-as-a-service that can be automatically deployed on public clouds, private clouds or bare-metal servers. As part of BiobankCloud, we also provide a Laboratory Information Management Service (LIMS) as software-as-a-service (SaaS). The LIMS has an integrated User Interface (UI) for authenicating/authorizing users, managing data, designing and searching for metadata, and support for running workflows and analysis jobs on Hadoop. The LIMS hides much of the complexity of the Hadoop backend, and supports multi-tenancy through first-class support for \textit{Studies}, \textit{SampleCollections} (DataSets), \textit{Samples}, and \textit{Users}.

In BiobankCloud, we have developed our own Hadoop distribution, Hadoop Open Platform-as-a-Service (Hops), with improved scalability and customizability properties, enabled by a new metadata storage system based on a distributed in-memory database.

Existing scientific workflow management systems are typically custom-built and have not been designed for data parallel processing with data locality. For performance reasons, their architectures and file formats are typically flat (monolithic). None of the main file formats in genomics (fastq, BAM/SAM, CRAM, and VCF) have support for data parallel processing because of their assumption of centralized metadata (e.g., in the file header). 
In BiobankCloud, we provide a scientific workflow management system, SaasFee, as a native YARN/Hadoop 2nd-level scheduler that provides a bridge between support for existing file formats and data parallel processing. SaasFee can both speedup workflows, such as a ??x speedup for NGS varient calling on XX machines, and scale-out to support larger clusters.

\section{Related work}
There are a couple of frameworks for data parallel processing of genomic data. Adam, Halvade, Seal, PigSeq, Spork?

There's the big Alvados? project in Harvard.

In security, Hadoop has support for Apache Ranger (attribute-based access control), Apache Sentry (Databases, RBAC), and Apache ?? (RBAC for REST APIs).

Nothing happening in Biobanking?



\section{Conclusions}
We have shown that a reliable leader election service can be implemented using two phase commit transactions in the NDB storage engine, a NewSQL database. Our solution ensures that there is never more than one leader, and the time taken for leader election is comparable to ZooKeeper for clusters of up to 800 processes. Our algorithm enables distributed systems that already use NewSQL databases to save the operational overhead of deploying a third-party service, such as ZooKeeper, for leader election, as our algorithm can easily be re-implemented for other NewSQL databases. 
\section{Acknowledgements}
This work funded by the EU FP7 project ``Scalable, Secure Storage and Analysis of Biobank Data'' under Grant Agreement no. 317871. 
\bibliographystyle{ieeetr}
\bibliography{bbc}

\end{document}
